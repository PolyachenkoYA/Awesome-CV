%-------------------------------------------------------------------------------
%	SECTION TITLE
%-------------------------------------------------------------------------------
\cvsection{Experience}


%-------------------------------------------------------------------------------
%	CONTENT
%-------------------------------------------------------------------------------
\begin{cventries}

%---------------------------------------------------------
  \cventry
    {Undergraduate student} % Job title
    {MIPT, \href{http://en.smcmp.ru/}{Laboratory of supercomputer methods
in condensed matter physics}} % Organization
    {Moscow, Russia} % Location
    {Sep. 2018 - Present} % Date(s)
    {
      \begin{cvitems} % Description(s) of tasks/responsibilities
        \item {\textsf{Mentored 4} \textsf{undergraduates} in their voluntary individual projects. 09.2020 -- present}
        \item {Computationally reproduced an experiment of measuring the bulk modulus for the \textsf{crystalline Lysozyme}. Supervisor - Stegailov V.V. 2020.}
        \item {Investigated behavior of the \textsf{Lennard-Jones system} near the boiling points via space-time correlators. Delivered reports at \textsf{several conferences}. The project was supported by the Russian Science Foundation. Supervisor - Norman G.E.. 2019.}
      	\item {Studied self-diffusion in Lennard-Jones system using classical MD implemented in \textsf{LAMMPS}. Delivered a report on the obtained results at the MIPT conference. Supervisors - Timofeev A.V. and Norman G.E. 2018.}
      	\item {\href{https://github.com/PolyachenkoYA/molecules}{Created} from scratch an MD simulation engine (\textsf{C/C++, CUDA, OpenMP, Python, Matlab}). The package was used to test and improve Kinetic Theory of Gases.  2018.}
      \end{cvitems}
    }
    
%---------------------------------------------------------
  \cventry
    {Summer Research Program Intern, * Remote due to COVID} % Job title
    {École Polytechnique Fédérale de Lausanne (\href{https://www.epfl.ch/en/}{EPFL}), \hspace{5pt}  \href{https://www.epfl.ch/labs/lbm/}{Laboratory for Biomolecular Modeling}} % Organization
    {* Lausanne, Switzerland} % Location
    {Jul. - Aug. 2020} % Date(s)
    {
      \begin{cvitems} % Description(s) of tasks/responsibilities
		\item{Suggested a method of defining an interface site for an unbound conformation using the MD trajectory of a bound complex.}
        \item {Benchmarked the \href{https://www.nature.com/articles/s41592-019-0666-6}{MaSIF-site} method on solvated protein conformations using the suggested method.}
      \end{cvitems}
    }
    
%---------------------------------------------------------    

%---------------------------------------------------------
  \cventry
    {Teaching Assistant} % Job title
    {MIPT, \hspace{5pt} Department of Computer Science} % Organization
    {Moscow, Russia} % Location
    {Sep. - Dec. 2019} % Date(s)
    {
      \begin{cvitems} % Description(s) of tasks/responsibilities
        \item {Worked as a mentor and as a teaching assistant on a Python CS freshmen course. Helped to design \href{http://cs.mipt.ru/python}{exercises} for the course.}
      \end{cvitems}
    }
    
%---------------------------------------------------------

  \cventry
    {Programmer} % Job title
    {Innovative Oil \& Gas Technologies (\href{http://www.iogt.ru/eng/}{IOGT}), \textit{D. of methodological support for geophysical well logging}} % Organization
    {Moscow, Russia} % Location
    {Sep. 2017 - Jun. 2021} % Date(s)
    {
      \begin{cvitems} % Description(s) of tasks/responsibilities
        \item {I participate in developing and supporting a number of subroutines for interpreting well-logging data.}
      \end{cvitems}
    }

%---------------------------------------------------------

  \cventry
    {Intern} % Job title
    {MIPT, \hspace{5pt} \textit{Laboratory of Mechanical Systems and Processes Modeling}} % Organization
    {Moscow, Russia} % Location
    {Aug. - Oct. 2018} % Date(s)
    {
      \begin{cvitems} % Description(s) of tasks/responsibilities
        \item {\href{https://github.com/PolyachenkoYA/geo}{Simulated} elastic wave propagation using ray tracing (Matlab, С/С++, OpenMP).}
        \item {The project was supported by the Russian Science Foundation.}
%        Разработка методики моделирования процессов, протекающих в теле человека %при применении интеллектуальных систем неинвазивной хирургии
      \end{cvitems}
    }

%---------------------------------------------------------
\end{cventries}
