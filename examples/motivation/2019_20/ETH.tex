%!TEX TS-program = xelatex
%!TEX encoding = UTF-8 Unicode
% Awesome CV LaTeX Template for Cover Letter
%
% This template has been downloaded from:
% https://github.com/posquit0/Awesome-CV
%
% Authors:
% Claud D. Park <posquit0.bj@gmail.com>
% Lars Richter <mail@ayeks.de>
%
% Template license:
% CC BY-SA 4.0 (https://creativecommons.org/licenses/by-sa/4.0/)
%


%-------------------------------------------------------------------------------
% CONFIGURATIONS
%-------------------------------------------------------------------------------
% A4 paper size by default, use 'letterpaper' for US letter
\documentclass[12pt, a4paper]{awesome-cv}

\usepackage{hyperref}
\hypersetup{
    colorlinks=true,
    linkcolor=blue,
    filecolor=magenta,      
    urlcolor=cyan,
}



% Configure page margins with geometry
\geometry{left=1.4cm, top=.8cm, right=1.4cm, bottom=1.8cm, footskip=.5cm}

% Specify the location of the included fonts
\fontdir[fonts/]

% Color for highlights
% Awesome Colors: awesome-emerald, awesome-skyblue, awesome-red, awesome-pink, awesome-orange
%                 awesome-nephritis, awesome-concrete, awesome-darknight
\colorlet{awesome}{awesome-red}
% Uncomment if you would like to specify your own color
% \definecolor{awesome}{HTML}{CA63A8}

% Colors for text
% Uncomment if you would like to specify your own color
% \definecolor{darktext}{HTML}{414141}
% \definecolor{text}{HTML}{333333}
% \definecolor{graytext}{HTML}{5D5D5D}
% \definecolor{lighttext}{HTML}{999999}

% Set false if you don't want to highlight section with awesome color
\setbool{acvSectionColorHighlight}{true}

% If you would like to change the social information separator from a pipe (|) to something else
\renewcommand{\acvHeaderSocialSep}{\quad\textbar\quad}


%-------------------------------------------------------------------------------
%	PERSONAL INFORMATION
%	Comment any of the lines below if they are not required
%-------------------------------------------------------------------------------
% Available options: circle|rectangle,edge/noedge,left/right
\photo[circle,edge,left]{me.jpg}
\name{Polyachenko}{Yury}
\position{Intern for the ETH Amgen program}
\address{42-8, Bangbae-ro 15-gil, Seocho-gu, Seoul, 00681, Rep. of KOREA}

\name{POLYACHENKO}{Yury}
%\position{Intern for EPFL E3 summer program}
\address{Krasnogo Mayaka Street 13А, b. 6, Moscow, Russia, 117570}

\mobile{+7(903)531-34-25}
\email{polyachenko.yua@phystech.edu}
%\homepage{www.posquit0.com}
\github{polyachenkoya}
\linkedin{polyachenkoya}
% \gitlab{gitlab-id}
% \stackoverflow{SO-id}{SO-name}
% \twitter{@twit}
\skype{polyachenkoya}
% \reddit{reddit-id}
% \medium{madium-id}
% \googlescholar{googlescholar-id}{name-to-display}
%% \firstname and \lastname will be used
% \googlescholar{googlescholar-id}{}
% \extrainfo{extra informations}

%\quote{``Be the change that you want to see in the world."}


%-------------------------------------------------------------------------------
%	LETTER INFORMATION
%	All of the below lines must be filled out
%-------------------------------------------------------------------------------
% The company being applied to
%\recipient
%  {Company Recruitment Team}
%  {Google Inc.\\1600 Amphitheatre Parkway\\Mountain View, CA 94043}
% The date on the letter, default is the date of compilation
%\letterdate{\today}
% The title of the letter
%\lettertitle{Job Application for Software Engineer}
% How the letter is opened
\letteropening{Dear Mr./Ms./Dr. LastName,}
% How the letter is closed
\letterclosing{Sincerely,}
% Any enclosures with the letter
%\letterenclosure[Attached]{Curriculum Vitae}


%-------------------------------------------------------------------------------
\begin{document}

% Print the header with above personal informations
% Give optional argument to change alignment(C: center, L: left, R: right)
\makecvheader[C]

% Print the footer with 3 arguments(<left>, <center>, <right>)
% Leave any of these blank if they are not needed
\makecvfooter
  {\thepage}
  {Polyachenko Yury~~~·~~~Statement of purpose}
  {\thepage}

% Print the title with above letter informations
%\makelettertitle
%\vspace{5pt}
%\hspace{5pt} 

%-------------------------------------------------------------------------------
%	LETTER CONTENT
%-------------------------------------------------------------------------------
\begin{cvletter}

\lettersection{About Me}
My name is Yury Polyachenko. I am a 3rd-year undergraduate student at the Moscow Institute of Physics and Technology (MIPT). As a freshman, I joined Prof. Genri Norman's lab of Computational Condensed Matter Physics and Life Systems. Now, I have been working there for over 2 years. I am currently investigating a discontinuity phenomenon in a stable-metastable phase transition in the Lennard-Jones system using LAMMPS and Matlab. Last summer, I attended the CECAM SISSA Molecular Dynamics (MD) school, where I studied polymer and protein dynamics and was fascinated by Prof. Mark Tuckerman's lectures on applying machine learning (ML) and MD to the drug design. Being intrigued, I completed courses on MD, high performance computing (HPC), and ML at MIPT. I want to continue exploring physical phenomena in biological systems theoretically and numerically. I believe ETH provides an excellent opportunity to challenge myself with more bio-oriented yet related to my background problems. Therefore, I am interested in joining several scientific groups in which I believe I can fully realize my scientific potential.

\lettersection{Laboratory for Movement Biomechanics}
The vast majority of master's projects offered in the lab employ Matlab, in which I am fluent. I believe my solid background in theoretical mechanics will be useful since most of the projects deal with mechanical macro movement. Specifically, I am excited by Rosa's project on ML for analysis of walking. Besides Matlab, I have some experience in Python, ML, and ML in Python, which I believe to be crucial skills for the project. I am interested in ML not only in a coding context but on a more fundamental level. For instance, while working on a final project on the ML course at MIPT, I came up with an idea of using generalized tensors of inertia as robust features for learning. That turned out to be close to the Moment Tensor Potentials. That required fluency in theoretical mechanics and a more holistic and sophisticated approach than needed in regular ML exercises commonly offered on introductory ML courses.

\lettersection{Computational Models of Morphogenesis}
I have a solid background in theoretical mechanics, differential equations, and computational mathematics. On the physical side, during my 6th semester, I am attending a "Soft Matter Physics" course, which will focus on membrane and tissue dynamics. Therefore, I think my background will quite fit the lab's LBIBCell project, and it will be fascinating for me to work on. So far, I have dealt with a multiscale approach, and it often can be useful to process boundary conditions in fluid dynamics. Those are one of the key points of the LBIBCell project. My first research project was about diffusion, which is another crucial piece of the lab's project. Finally, the possibility of 3D generalization of the project and emerging optimization issues are mentioned. For these problems, I believe I can make use of my mathematical background and HPC skills in C++.

\end{cvletter}

%-------------------------------------------------------------------------------
% Print the signature and enclosures with above letter informations
To sum up, I am immensely interested in joining ETH Amgen as a summer intern. The program would provide an excellent opportunity to develop both professionally and academically, which I believe would prove critical when applying for a doctoral degree program at ETH. I thank you for your time, and I genuinely appreciate your consideration of my application. Please do not hesitate to contact me at \newline \href{mailto:polyachenko.yua@phystech.edu}{polyachenko.yua@phystech.edu}.

Sincerely,   Yury POLYACHENKO.

%\makeletterclosing

\end{document}

