%!TEX TS-program = xelatex
%!TEX encoding = UTF-8 Unicode
% Awesome CV LaTeX Template for Cover Letter
%
% This template has been downloaded from:
% https://github.com/posquit0/Awesome-CV
%
% Authors:
% Claud D. Park <posquit0.bj@gmail.com>
% Lars Richter <mail@ayeks.de>
%
% Template license:
% CC BY-SA 4.0 (https://creativecommons.org/licenses/by-sa/4.0/)
%


%-------------------------------------------------------------------------------
% CONFIGURATIONS
%-------------------------------------------------------------------------------
% A4 paper size by default, use 'letterpaper' for US letter
\documentclass[11pt, a4paper]{awesome-cv}

\usepackage{hyperref}
\hypersetup{
    colorlinks=true,
    linkcolor=blue,
    filecolor=magenta,      
    urlcolor=cyan,
}



% Configure page margins with geometry
\geometry{left=1.4cm, top=.8cm, right=1.4cm, bottom=1.8cm, footskip=.5cm}

% Specify the location of the included fonts
\fontdir[fonts/]

% Color for highlights
% Awesome Colors: awesome-emerald, awesome-skyblue, awesome-red, awesome-pink, awesome-orange
%                 awesome-nephritis, awesome-concrete, awesome-darknight
\colorlet{awesome}{awesome-red}
% Uncomment if you would like to specify your own color
% \definecolor{awesome}{HTML}{CA63A8}

% Colors for text
% Uncomment if you would like to specify your own color
% \definecolor{darktext}{HTML}{414141}
% \definecolor{text}{HTML}{333333}
% \definecolor{graytext}{HTML}{5D5D5D}
% \definecolor{lighttext}{HTML}{999999}

% Set false if you don't want to highlight section with awesome color
\setbool{acvSectionColorHighlight}{true}

% If you would like to change the social information separator from a pipe (|) to something else
\renewcommand{\acvHeaderSocialSep}{\quad\textbar\quad}


%-------------------------------------------------------------------------------
%	PERSONAL INFORMATION
%	Comment any of the lines below if they are not required
%-------------------------------------------------------------------------------
% Available options: circle|rectangle,edge/noedge,left/right
%\photo[circle,noedge,left]{./examples/profile}
\name{Polyachenko}{Yury}
\position{Intern for the EPFL E3 summer program}
\address{42-8, Bangbae-ro 15-gil, Seocho-gu, Seoul, 00681, Rep. of KOREA}

\name{Polyachenko}{Yury}
%\position{Intern for EPFL E3 summer program}
\address{Krasnogo Mayaka Street 13А, b. 6, Moscow, Russia, 117570}

\mobile{+7(903)531-34-25}
\email{polyachenko.yua@phystech.edu}
%\homepage{www.posquit0.com}
\github{polyachenkoya}
%\linkedin{polyachenkoya}
% \gitlab{gitlab-id}
% \stackoverflow{SO-id}{SO-name}
% \twitter{@twit}
\skype{polyachenkoya}
% \reddit{reddit-id}
% \medium{madium-id}
% \googlescholar{googlescholar-id}{name-to-display}
%% \firstname and \lastname will be used
% \googlescholar{googlescholar-id}{}
% \extrainfo{extra informations}

%\quote{``Be the change that you want to see in the world."}


%-------------------------------------------------------------------------------
%	LETTER INFORMATION
%	All of the below lines must be filled out
%-------------------------------------------------------------------------------
% The company being applied to
\recipient
  {Company Recruitment Team}
  {Google Inc.\\1600 Amphitheatre Parkway\\Mountain View, CA 94043}
% The date on the letter, default is the date of compilation
\letterdate{\today}
% The title of the letter
\lettertitle{Job Application for Software Engineer}
% How the letter is opened
\letteropening{Dear Mr./Ms./Dr. LastName,}
% How the letter is closed
\letterclosing{Sincerely,}
% Any enclosures with the letter
\letterenclosure[Attached]{Curriculum Vitae}


%-------------------------------------------------------------------------------
\begin{document}

% Print the header with above personal informations
% Give optional argument to change alignment(C: center, L: left, R: right)
\makecvheader[C]

% Print the footer with 3 arguments(<left>, <center>, <right>)
% Leave any of these blank if they are not needed
\makecvfooter
  {\thepage}
  {Polyachenko Yury~~~·~~~Statement of purpose}
  {\thepage}

% Print the title with above letter informations
%\makelettertitle
\vspace{25pt}
\hspace{5pt} Dear Prof. Smit, Prof. Curtin, Prof. Marzari and other members of the selection committee, 

%-------------------------------------------------------------------------------
%	LETTER CONTENT
%-------------------------------------------------------------------------------
\begin{cvletter}

\letterintrosection

\hspace{10pt} My name is Yury Polyachenko and I am a third-year undergraduate student at MIPT - one of the most famous and prestigious universities in physical sciences in Russia. I am applying for the EPFL E3 program because at EPFL I've found hosting labs which in my view correlate with my interests and background astoundingly well. Therefore here is my a bit structured tail about myself and why I think me and EPFL are a grate match:

\lettersection{About Me}
I've been fascinated about the idea of computer simulation since 9$^{th}$ school grade. At a particular moment back then I realized that it seemed I'd known enough to try to see how a system moves according to Newton's law without solving equations exactly as we always did in school. I spent next 2 years (10$^{th}$ and 11$^{th}$ grades) under the supervision of a head of a lab in the Institute of Astronomy RAS. I was modeling stellar dynamics in a galaxy and mastering various computer skills along the way. Because of my school achievements I could go to any Russian university with no exams required. I was leaning towards applied and computational physics and MIPT had the best Physics department in the country so that defined my choice. 

At the MIPT I continued to develop my computer skills and knowledge of physics. Regular courses success is represented by the GPA 4.99/5. My more specific interest still only grew and so did my skills. I say that because of a summer internship offer I got after my freshmen year from my CS seminarian who worked at MIPT in the <<Laboratory of Mechanical Systems and Processes Modeling>>. I've created a proof-of-concept model which was used for further studies. 

At the same time in the beginning of my 2$^{nd}$ semester I found a lab where I though I wanted to try to dig deeper. The head of the lab was Genry Norman with whom I've been working for 2 years now. My first project where was devoted to simulation of self-diffusion in the Lerrand-Jones system. At the end of the year the project was elected top 10 of the class which consisted of $\sim$ 1100 people. I spent the 1$^{st}$ half of my sophomore year improving the project and preparing an oral report for the MIPT conference. The report was again a success -- I won 2$^{nd}$ place in the section among bachelors. After the exams I was occupied by nothing but regular studies. It happened so that a programmer position at the company my other CS seminarian was working in became available and I got accepted. During the work at IOGT I significantly improved my matlab skills because it was the main language. Before the end of the academic year I was accepted for the CECAM summer school at SISSA. In my 4$^{th}$ semester, right before the school, I'd completed a Molecular Dynamics course and got the best final grade in the class. So the summer school was a nice continuation and repetition of the completed course. Mark Tuckerman gave a series of lectures on that school and I really liked him and the topics he covered. I really enjoyed communicating with students from different cultures and specializations and explaining difficult moments to them. That gave me an idea to try teaching students. I knew that in Russia there is an educational center called Sirius where I can apply with a project and will guide a group of high school students if accepted. I applied with a new task I got from Norman by that time. I spent next month helping students master basic computer skills, physics and math essential for understanding and performing molecular dynamics. In the end they managed to run a simple simulation on  a remote server in LAMMPS and to verify basic laws such as energy conservation or maxwell distribution. At that moment I liked working with those kids because they were really motivated and eager for new things. I continued to work on the project and delivered an oral report in the New Athos in the August. Remembering the positive experience with kids I decided to try to teach freshmen. They did a new CS python course and I was relatively good in it, so that was it. For next 4 months I've been doing my research work and helping freshmen with python exercises. Averaged over longer period and over not so united group average motivation wasn't so high so I can't say I loved every moment of the process but it was an interesting experience. Updated project was presented on the 62$^{th}$ MIPT conference but I yet continued to work on it. At that point it was not only me but another 3$^{rd}$-year student and a PhD student. Our final results will be presented on the international ELBRUS 2020 conference. With Norman we are also working on an article. I hope now you have a certain understanding of my interests and my background. Thus I can move to questions <<why me for EPFL?>> and <<why EPFL for me?>>

\lettersection{Why Google?}
Suspendisse commodo, massa eu congue tincidunt, elit mauris pellentesque orci, cursus tempor odio nisl euismod augue. Aliquam adipiscing nibh ut odio sodales et pulvinar tortor laoreet. Mauris a accumsan ligula. Class aptent taciti sociosqu ad litora torquent per conubia nostra, per inceptos himenaeos. Suspendisse vulputate sem vehicula ipsum varius nec tempus dui dapibus. Phasellus et est urna, ut auctor erat. Sed tincidunt odio id odio aliquam mattis. Donec sapien nulla, feugiat eget adipiscing sit amet, lacinia ut dolor. Phasellus tincidunt, leo a fringilla consectetur, felis diam aliquam urna, vitae aliquet lectus orci nec velit. Vivamus dapibus varius blandit.

\lettersection{Why Me?}
Duis sit amet magna ante, at sodales diam. Aenean consectetur porta risus et sagittis. Ut interdum, enim varius pellentesque tincidunt, magna libero sodales tortor, ut fermentum nunc metus a ante. Vivamus odio leo, tincidunt eu luctus ut, sollicitudin sit amet metus. Nunc sed orci lectus. Ut sodales magna sed velit volutpat sit amet pulvinar diam venenatis.

\end{cvletter}


%-------------------------------------------------------------------------------
% Print the signature and enclosures with above letter informations

I guess the thing that was exiting me back when and still does in doing a physical simulations is that personally for me it has just right ratio of theory and practice in it. Also I like to see the result of what I am doing and computational physics provides me with such opportunity.

\makeletterclosing

\end{document}
