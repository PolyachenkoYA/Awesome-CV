%!TEX TS-program = xelatex
%!TEX encoding = UTF-8 Unicode
% Awesome CV LaTeX Template for Cover Letter
%
% This template has been downloaded from:
% https://github.com/posquit0/Awesome-CV
%
% Authors:
% Claud D. Park <posquit0.bj@gmail.com>
% Lars Richter <mail@ayeks.de>
%
% Template license:
% CC BY-SA 4.0 (https://creativecommons.org/licenses/by-sa/4.0/)
%


%-------------------------------------------------------------------------------
% CONFIGURATIONS
%-------------------------------------------------------------------------------
% A4 paper size by default, use 'letterpaper' for US letter
\documentclass[12pt, a4paper]{awesome-cv}

\usepackage{hyperref}
\hypersetup{
    colorlinks=true,
    linkcolor=blue,
    filecolor=magenta,      
    urlcolor=cyan,
}



% Configure page margins with geometry
\geometry{left=1.4cm, top=.8cm, right=1.4cm, bottom=1.8cm, footskip=.5cm}

% Specify the location of the included fonts
\fontdir[fonts/]

% Color for highlights
% Awesome Colors: awesome-emerald, awesome-skyblue, awesome-red, awesome-pink, awesome-orange
%                 awesome-nephritis, awesome-concrete, awesome-darknight
\colorlet{awesome}{awesome-red}
% Uncomment if you would like to specify your own color
% \definecolor{awesome}{HTML}{CA63A8}

% Colors for text
% Uncomment if you would like to specify your own color
% \definecolor{darktext}{HTML}{414141}
% \definecolor{text}{HTML}{333333}
% \definecolor{graytext}{HTML}{5D5D5D}
% \definecolor{lighttext}{HTML}{999999}

% Set false if you don't want to highlight section with awesome color
\setbool{acvSectionColorHighlight}{true}

% If you would like to change the social information separator from a pipe (|) to something else
\renewcommand{\acvHeaderSocialSep}{\quad\textbar\quad}


%-------------------------------------------------------------------------------
%	PERSONAL INFORMATION
%	Comment any of the lines below if they are not required
%-------------------------------------------------------------------------------
% Available options: circle|rectangle,edge/noedge,left/right
\photo[circle,edge,left]{me.jpg}

\name{Поляченко}{Юрий}
\position{\underline{Позиция}: Студент международной Суперкомпьютерной Академии}
\address{4 курс ЛФИ \textbf{ФОПФ}/ФПФЭ МФТИ, ОП ВФКСиЖС}

\mobile{+7(903)531-34-25}
\email{polyachenko.yua@phystech.edu}
%\homepage{www.posquit0.com}
\github{polyachenkoya}
\linkedin{polyachenkoya}
% \gitlab{gitlab-id}
% \stackoverflow{SO-id}{SO-name}
% \twitter{@twit}
\skype{polyachenkoya}
% \reddit{reddit-id}
% \medium{madium-id}
% \googlescholar{googlescholar-id}{name-to-display}
%% \firstname and \lastname will be used
% \googlescholar{googlescholar-id}{}
% \extrainfo{extra informations}

%\quote{``Be the change that you want to see in the world."}


%-------------------------------------------------------------------------------
%	LETTER INFORMATION
%	All of the below lines must be filled out
%-------------------------------------------------------------------------------
% The company being applied to
%\recipient
%  {Company Recruitment Team}
%  {Google Inc.\\1600 Amphitheatre Parkway\\Mountain View, CA 94043}
% The date on the letter, default is the date of compilation
%\letterdate{\today}
% The title of the letter
%\lettertitle{Job Application for Software Engineer}
% How the letter is opened
\letteropening{Dear Mr./Ms./Dr. LastName,}
% How the letter is closed
\letterclosing{Sincerely,}
% Any enclosures with the letter
%\letterenclosure[Attached]{Curriculum Vitae}


%-------------------------------------------------------------------------------
\begin{document}

% Print the header with above personal informations
% Give optional argument to change alignment(C: center, L: left, R: right)
\makecvheader[C]

% Print the footer with 3 arguments(<left>, <center>, <right>)
% Leave any of these blank if they are not needed
\makecvfooter
  {\thepage}
  {Поляченко Юрий~~~·~~~Мотивационное письмо}
  {\thepage}

% Print the title with above letter informations
%\makelettertitle
%\vspace{5pt}
%\hspace{5pt} 

%-------------------------------------------------------------------------------
%	LETTER CONTENT
%-------------------------------------------------------------------------------
\begin{cvletter}

\lettersection{Обо мне}
Меня зовут Поляченко Юрий, сейчас я окончил 3 курса бакалавриата ФОПФ МФТИ на образовательной программе <<Выч. физика конд. сост. и живых систем>>. Я интересовался физическим  моделированием и вычислениями с 10 класса. С тех пор я занимался N-body моделированием формирования спиральной структуры галактик из равномерного диска, а так же переносом нейтронов в кусочно-однородной среде (аля MCNP). 

В МФТИ под куроводством Нормана Генри Эдгаровича я исследовал Леннард-Джонсовскую систему как с вычислительной так и с физической точек зрения. Я писал различные схемы распараллеливания Л-Дж на CPU (OpenMP, MPI) и на GPU (CUDA). С помощью пакета LAMMPS я исследовал микроскопически-коллективное поведение Л-Дж частиц вблизи кривой фазового равновестия. Работа докладывалась на международной конференции ELBRUS2020 и сейчас under review в PRL. 

\lettersection{Текущая работа}
Сейчас под руководством Стегайлова Владимира Владимировича я с использованием GROMACS занимаюсь in-silico воспроизведением эксперимента с кристаллическим белком. Задача является вычислительно сложной, т.к. необходимо исследовать взаимосвязь физических параметров, для чего необходимо делать расчеты по многомерным сеткам параметров, а система белок-вода может быть заметно неоднородной и к тому же содержать большое количество атомов. Данные еще набираются, но уже на текущем этапе 1 траектория может занимать до 30 Гб в формате xtc.

Так же на стажировке в EPFL в лаборатории \href{https://www.epfl.ch/labs/lbm/}{LBM} я работаю над проверкой применимости метода MaSIF-site для докинга белков в воде. Каждая траектория, содержащая 50,000 дампов и занимающая $\sim 5$ Гб в xtc, требует попарного анализа кадров, что создает вычислительно-сложную задачу.

Пост-обработка результатов в обоих проектах происходит с использованием python.

\lettersection{Академия}
В связи с моими интересами и задачами, стоящими в моей текущей работе, я хочу пройти трек <<Python для HPC>>. Я думаю, программа данного трека поможет мне эффективнее реализовывать обработку данных на python в моих текущих работах. К тому же, я планирую продолжать исследования в области вычислитеной биофизики, а в этой области python является общепринятым стандартом как обработки, так и часто даже генерации данных, поэтому я считаю, что изучание HPC-стороны этого языка поможет мне в целом в проффесиональном развитии.

\end{cvletter}

С уважением, Поляченко Юрий

%\makeletterclosing

\end{document}

